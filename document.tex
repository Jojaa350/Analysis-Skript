% !TeX spellcheck = de_DE
\documentclass[halfparscip]{scrartcl}

%German language
\usepackage[utf8]{inputenc}
\usepackage[ngerman]{babel}
\usepackage[T1]{fontenc}
%Math
\usepackage[fleqn]{amsmath}
\usepackage{amssymb, amstext}
\usepackage{mathtools}
\usepackage{ifsym}
\usepackage{xfrac}
%Units
\usepackage[binary-units]{siunitx}  
\sisetup{locale = DE,per-mode=fraction} 
%Logo
\usepackage[absolute]{textpos}
\usepackage{graphicx}
%Better enumeration
\usepackage{enumitem}
%Better tables
\usepackage{tabularx}
%Graphs and trees
\usepackage{tikz}
\usetikzlibrary{trees}
%Code
\usepackage{listings}
\usepackage{lstautogobble}
\usepackage{verbatim}
\usepackage{fancyvrb}
%Links
\usepackage{xcolor}
\usepackage{hyperref}
%For redefining tableofcontents to be of black links
\usepackage{letltxmacro}


%Define colors of links
\definecolor{urlblue}{cmyk}{0.7, 0.3, 0, 0.100}
\definecolor{linkblue}{cmyk}{0.7, 0.3, 0, 0.250}
\hypersetup{colorlinks=true, urlcolor={urlblue}, linkcolor=linkblue, allbordercolors=white}

\RedeclareSectionCommand[beforeskip=1.3cm]{section} %increase distance to next section

%Declare further math operators
\newcommand{\join}[2]{{\tiny \; \textifsym{|><|}}\,_{#1=#2}}
\newcommand*\dif{\mathop{}\!\mathrm{d}}
\DeclareMathOperator*{\bigtimes}{\mathop{\raisebox{-.5ex}{\hbox{\huge{$\times$}}}}}
\DeclareMathOperator{\nequiv}{\not\equiv}
\newcommand{\smallvector}[2]{{\big (}{\tiny \begin{matrix} #1\\ #2 \end{matrix}}{\big )}}

\lstset {
	language=C++,
	linewidth=100ex, % fits 88 characters 
	escapechar=§,
	emph={uint32\_t, uint64\_t},
	literate={~}{{\footnotesize$\sim$}}1,
	autogobble=true,
	keywordstyle=\bfseries\ttfamily\color[rgb]{0,0,1},
	emphstyle=\bfseries\ttfamily\color[rgb]{0,0,1},
	identifierstyle=\ttfamily,
	commentstyle=\ttfamily\color[rgb]{0.133,0.545,0.133},
	stringstyle=\ttfamily\color[rgb]{0.627,0.126,0.941},
	showstringspaces=false,
	basicstyle=\ttfamily\small,
	numberstyle=\footnotesize,
	numbers=left,
	stepnumber=1,
	numbersep=10pt,
	tabsize=4,
	breaklines=true,
	postbreak = \raisebox{0ex}[0ex][0ex]{\ensuremath{\color{red}\hookrightarrow\space}},
	breakatwhitespace=true,
	aboveskip={\baselineskip},
}

%Makes Links in Table of contents black.
\LetLtxMacro{\oldtableofcontents}{\tableofcontents}
\renewcommand{\tableofcontents}{\hypersetup{linkcolor=black}\oldtableofcontents\hypersetup{linkcolor=linkblue}}

\newcounter{subsection2}

% ----------------------------------------------------------------------------------------------------------------------

\title{Analysis für Informatik}
\subtitle{MA0902}
\author{Josef Schönberger}
\date{\today}

%\setcounter{section}{0}

\begin{document}
%Logo
\begin{textblock*} {120mm}(150mm,5mm) % first number for hight, next two for position
	\includegraphics[width=0.4\textwidth]{tum_logo}
\end{textblock*}

\maketitle
\tableofcontents
\newpage

\section{Mengenterminologie}
\subsection{Teilmenge}
$B \subseteq A$ gdw. $\forall x \in B$ gilt auch, dass $x \in A$.

\subsection{Injektiv / Surjektiv}
$f: A \rightarrow B$ ist
\begin{itemize}
	\item injektiv, falls $\forall x, y \in A$ mit $x \neq y$ gilt: $f(x) \neq f(y)$
	\item surjektiv, falls $\forall y \in B$ gilt: $\exists x \in A$, sodass $f(x) = y$
	\item bijektiv, falls sowohl injektiv und surjektiv
\end{itemize}

\subsection{Auswahl-Axiom}
$\exists$ Surjektion von $A$ nach $B$ gdw. $\exists$ Injektion von $B$ nach $A$

\subsection{Exklusivität des Vergleichs der Kardinalität}
$\forall A, B$ Mengen: Entweder $\vert A\vert \leq \vert B \vert$ oder $\vert A \vert \geq \vert B \vert$

\subsection{Cantor-Bernstein}
Falls Injektionen $f: A \rightarrow B$ und $g: B \rightarrow A$ existieren, so gibt es eine Bijektion zwischen $A$ und $B$.

\subsection{Abzählbarkeit}
Menge $A$ heißt abzählbar, falls $\vert A \vert = \vert \mathbb{N} \vert$.\\
$\aleph_0 = \vert \mathbb{N}\vert$

\newpage
\section{Reelle Zahlen und Vektoren}
\subsection{Körper und Anordnung}
Ein Körper $(F,+,\cdot)$ ist eine Menge $F$ mit den Operationen $+$ und $\cdot$, wobei:
\begin{enumerate}
	\item Addition assoziativ
	\item Addition kommutativ
	\item Es existiert ein zur Addition neutrales Element $0$
	\item Es existiert stets ein zur Addition inverses Element
	\item Multiplikation assoziativ
	\item Multiplikation kommutativ
	\item Es existiert ein zur Multiplikation neutrales Element $1$
	\item Es existiert stets ein zur Multiplikation inverses Element
	\item Es gilt das Distributionsgesetz
\end{enumerate}
%
Ein Körper heißt angeordnet, falls eine Relation $<$ existiert, sodass $\forall x, y \in F$:
\begin{enumerate}
	\item Entweder $x < y$ oder $x = y$ oder $x > y$
	\item Falls $x, y > 0$, dann $x + y, x \cdot y > 0$
	\item $x > y$ gdw. $x - y > 0$
\end{enumerate}

\subsection{Obere / Untere Schranke}
$x \in \mathbb{R}$ ist eine obere Schranke von $M \subseteq \mathbb{R}$, falls $\forall y \in M$ gilt: $y \leq x$.\\
Falls ein solches $x$ existiert, so heißt $M$ nach oben beschränkt, sonst nach oben unbeschränkt.\\
Vergleichbar ist die untere Schranke.

\subsection{Maximum / Minimum}
$x \in \mathbb{R}$ ist das Maximum von $M \subseteq \mathbb{R}$, falls $x$ obere Schranke von $M$ und $x \in M$.\\
Vergleichbar ist das Minimum.

\subsection{Supremum / Infimum}
Das Supremum $\sup(M)$ ist die niedrigste obere Schranke von $M \subseteq \mathbb{R}$. Falls nicht existent, so schreiben wir $\sup(M) = \infty$.\\
Das Infimum $\inf(M)$ ist die höchste untere Schranke von $M \subseteq \mathbb{R}$. Falls nicht existent, so schreiben wir $\inf(M) = -\infty$.

\subsection{Definition Vollständigkeit}
Ein angeordneter Körper $K$ ist vollständig, falls $\forall M \subseteq K$ nach oben beschränkte Teilmenge mit $M \neq \emptyset$ ein Supremum besitzt.

\subsection{Vollständigkeitsaxiom}
$\mathbb{R}$ ist vollständig.

\subsection{Rechenregeln}
Sein $A, B \in \mathbb{R}$ mit $\sup(A), \sup(B) \in \mathbb{R}$.
\begin{enumerate}
	\item $\sup(A + B) = \sup(A) + \sup (B)$\footnote{$A + B = \{a + b \mid a \in A, b \in B\}$}
	\item $\forall \lambda \geq 0: \sup(\lambda \cdot A) = \lambda \cdot \sup(A)$ \footnote{$\lambda \cdot A = \{\lambda \cdot a \mid a \in A\}$}
	\item Falls $A, B \subseteq [0,\infty)$: $\sup(A \cdot B) = \sup(A) \cdot \sup(B)$ \footnote{$A \cdot B = \{a \cdot b \mid a \in A, b \in B\}$}
	\item\label{Zahlen:Rechenregeln:subset} $A \subseteq B \ \Rightarrow \  \sup(A) \leq \sup(B)$
\end{enumerate}
Selbiges für das Infimum (wobei bei \autoref{Zahlen:Rechenregeln:subset} das '$\leq$' durch ein '$\geq$' zu ersetzen ist)

\subsection{Umgebung}
Ein offenes Intervall $(a,b)$ ist eine Umgebung von $x$, falls $x \in (a,b)$.

\subsection{Offenheit einer Menge}
$A \subseteq \mathbb{R}$ ist offen, falls $\forall x \in A$ gilt: $\exists I_x$ Umgebung von $x$, sodass $I_x \subseteq A$

\subsection{Abgeschlossenheit einer Menge}
$A \subseteq \mathbb{R}$ ist abgeschlossen, falls $\mathbb{R} \backslash A$ offen ist.

\subsection{Zeitgleich offen und abgeschlossen}
Nur $\mathbb{R}$ und $\emptyset$ sind offen und abgeschlossen.

\subsection*{Skalarprodukt}
\begin{align*}
	\cdot: \mathbb{R}^n \times \mathbb{R}^n &\rightarrow \mathbb{R}\\
	\overline{x} \cdot \overline{y} &\mapsto \sum_{k=1}^nx_ky_k\\
\end{align*}

Seien $\overline{x}, \overline{y}, \overline{z} \in \mathbb{R}^n$ und $\alpha \in \mathbb{R}$:
\begin{enumerate}
	\item $\overline{x} \cdot \overline{y} = \overline{y} \cdot \overline{x}$
	\item $(\alpha \overline{x}) \cdot \overline{y} = \overline{x} \cdot (\alpha \overline{y}) = \alpha(\overline{x} \cdot \overline{y})$
	\item $(\overline{x} + \overline{y}) \cdot \overline{z} = \overline{x}\cdot\overline{z} + \overline{y}\cdot\overline{z}$
	\item $\overline{x} \cdot \overline{x} \geq 0$
\end{enumerate}

\subsection*{Euklidische Norm}
\begin{equation*}
\vert\vert (x_1, \dots, x_n) \vert\vert := \sqrt{(x_1, \dots, x_n)\cdot(x_1, \dots, x_n)} = \sqrt{\sum_{k=1}^{n}x^2_k}
\end{equation*}

\subsection{Cauchy-Schwarz-Ungleichung}
$\forall \overline{x},\overline{y}\in\mathbb{R}^n$:
\begin{equation*}
	\left\vert\left<\overline{x},\overline{y}\right>\right\vert \leq \vert\vert\overline{x}\vert\vert \cdot \vert\vert\overline{y}\vert\vert\footnotemark
\end{equation*}\footnotetext{$\left<\overline{x},\overline{y}\right>$ beschreibt das Skalarprodukt von $\overline{x}$ und $\overline{y}$}
Gleichheit gilt gdw. $\overline{x}$ und $\overline{y}$ linear abhängig sind.

\subsection{Dreiecksungleichung}
$\forall \overline{x},\overline{y}\in\mathbb{R}^n$:
\begin{equation*}
	\vert\vert \overline{x} + \overline{y} \vert\vert \leq \vert\vert\overline{x}\vert\vert + \vert\vert\overline{y}\vert\vert
\end{equation*}

\subsection{Ungleichung geom. und arithm. Mittel}
Seien $x,y \geq 0$.
\begin{equation*}
	\sqrt{xy} \leq \frac{x + y}{2}
\end{equation*}

\subsection*{Komplexe Zahlen}
$i := \sqrt{-1}$.\\
$\mathbb{C} := \left\{x + yi \;\middle|\; x, y \in \mathbb{R}\right\}$\\\\
Seien $z_1 = x_1 + y_2i, z_2 = x_2 + y_2i \in \mathbb{C}$ komplexe Zahlen. Wir definieren:
\begin{itemize}
	\item die Addition: $z_1 + z_2 = (x_1 + x_2) + (y_1 + y_2)i$
	\item die Subtraktion: $z_1 - z_2 = (x_1 - x_2) + (y_1 - y_2)i$
	\item die Multiplikation: $z_1 \cdot z_2 = x_1x_2 - y_1y_2 + (x_1y_2 + x_2y_1)i$
	\item die Konjugierte: $\overline{z_1} = x_1 - y_1i$
	\item den Betrag: $\vert z_1\vert = \sqrt{x_1^2 + y_1^2}$
\end{itemize}

\newpage
\section{Folgen}

\subsection{Definition Folge}
Eine Folge mit Werten ist eine Abbildung $\mathbb{N}^+ \rightarrow M$, wobei $M$ eine beliebige Menge.\\
Wir schreiben $x_1, x_2, ...$.\\
Alternativ kann eine Folge auch mit $\mathbb{N}_0$ definiert sein.

\subsection{Definition (streng) monoton}
Eine Folge $(x_n)_n$ heißt monoton wachsend, falls $\forall n: x_n \leq x_{n+1}$.\\
Eine Folge $(x_n)_n$ heißt streng monoton wachsend, falls $\forall n: x_n < x_{n+1}$.\\\\
Vergleichbar definiert ist (streng) monoton fallend.

\subsection{Definition Grenzwert einer Folge}
$x \in \mathbb{R}$ heißt Grenzwert / Limes einer Folge $(x_n)_n$, falls:
\begin{equation*}
	\forall \varepsilon > 0:\;\; \exists N \in \mathbb{N}:\;\; \forall n \geq N:\;\; \vert x_n - x\vert < \varepsilon
\end{equation*}
Man schreibt:
\begin{align*}
	x_n &\xrightarrow[n \rightarrow \infty]{} x &&\text{oder} & x &= \lim_{n \rightarrow \infty} x_n
\end{align*}
$x_n$ heißt konvergent gdw. ein Grenzwert existiert.

\subsection{Eindeutigkeit eines Grenzwerts}
Jede reelle Folge hat max. einen Grenzwert.

\subsection{Beschränkung des Grenzwerts}
Falls $(x_n)_n \rightarrow x$ und $(y_n)_n \rightarrow y$ beschränkte Folgen mit $\forall n: \; x_n \leq y_n$, so gilt $x \leq y$.

\subsection{Einschließung}
Falls $(x_n)_n \rightarrow x$ und $(y_n)_n \rightarrow x$ beschränkte Folgen mit $\forall n: \; x_n \leq y_n$. 
Für jede weitere Folge $(w_n)$ mit $\forall n: \; x_n \leq w_n \leq y_n$ gilt dann: $w_n \rightarrow x$

\subsection{Definition Beschränktheit}
$(x_n)$ geht gegen $+\infty$, falls 
\begin{equation*}
	\forall C > 0: \;\; \exists N \in \mathbb{N}: \;\; \forall n \geq N: \;\; x_n \geq C
\end{equation*}
Vergleichbar geht $(x_n)$ gegen $-\infty$. \\
$(x_n)$ heißt hingegen beschränkt, falls $\exists K > 0$, sodass $\forall n : \;\; \vert x_n \vert \leq K$

\subsection{Beschränktheit durch Grenzwert}
Jede konvergente Folge ist beschränkt.

\subsection{Rechenregeln}
Seien $(x_n)$ und $(y_n)$ Folgen mit $x_n \rightarrow a$ und $y_n \rightarrow b$. Dann gilt:
\begin{itemize}
	\item $x_n + y_n \rightarrow a + b$
	\item $x_n - y_n \rightarrow a + b$
	\item $x_n \cdot y_n \rightarrow a \cdot b$
	\item $x_n / y_n \rightarrow a / b$, falls $b \neq 0$
\end{itemize}

\subsection{Konvergenz gegen Supremum / Infimum}
Jede monoton wachsende, beschränkte Folge konvergiert gegen ihr Supremum:
\begin{equation*}
	x = \lim_{n \rightarrow \infty} x_n = \sup\hspace{0cm}_n(x_n) := \sup(M) \;\; \text{ mit } M := \left\{x_n \;\vert\; x \in \mathbb{N}\right\}
\end{equation*}

\noindent Ebenso konvergiert jede monoton fallende, beschränkte Folge gegen ihr Infimum.

\subsection*{Definition Limes superior und Limes inferior}
Sei $(x_n)$ \underline{beschränkt}.
Dann ist definiert:
\begin{align*}
	\limsup_{n \rightarrow \infty} x_n &:= \lim_{n \rightarrow \infty} (\sup_{k \geq n} x_k) \\
	\liminf_{n \rightarrow \infty} x_n &:= \lim_{n \rightarrow \infty} (\inf_{k \geq n} n_k)
\end{align*}
Es folgt: $\displaystyle\liminf_{n \rightarrow \infty} x_n  \leq \limsup_{n \rightarrow \infty} x_n$

\subsection{Häufungspunkt}
Sei $(x_n)$ eine reellwertige Folge. \\
Seien $n_1 < n_2 < ... < n_k \in \mathbb{N}$ gegeben, so heißt $(x_{n_k})$ Teilfolge von $(x_n)$.\\
$x \in \mathbb{R}$ heißt Häufungspunkt, falls es eine Teilfolge mit $\lim\limits_{k \rightarrow \infty} x_{n_k} = x$ gibt.

\subsection{Bolzano-Weierstrass}
Falls $(x_n)$ und reellwertig:
\begin{equation*}
	\lim_{n \rightarrow \infty} x_n = x \text{ gdw. } \liminf_{n \rightarrow \infty} x_n = \limsup_{n \rightarrow \infty} x_n = \lim_{n \rightarrow \infty} x_n
\end{equation*}
Letztere Gleichung besagt, dass $\limsup\limits_{n \rightarrow \infty} x_n$ und $\liminf\limits_{n \rightarrow \infty} x_n$ zugleich maximaler und minimaler Häufungspunkt sind.\\\\
Diese Aussage ist äquivalent zu: Jede beschränkte Folge hat mindestens eine konvergente Teilfolge (da besagter Wert, auf den konvergiert wird, ein Häufungspunkt ist).

\subsection{Cauchys Kriterium für Konvergenz}
$(x_n)$ konvergiert gdw. 
\begin{equation*}
	\forall \varepsilon > 0:\;\; \exists N : \;\; \forall m,n > N: \;\; \vert x_n - x_m\vert < \varepsilon
\end{equation*}

\newpage
\section{Reihen}
\subsection{Definition Reihe}
Sei $a_n$ eine (komplexe) Folge. Die Folge 
\begin{equation*}
	s_n := a_0 + ... + a_n = \sum_{k=0}^{n}a_k
\end{equation*}
heißt unendliche Reihe (kurz: Reihe) mit den Gliedern $a_n$ und den Partialsummen $s_n$.\\\\
Falls $s_n$ konvergent:
\begin{equation*}
	s = \lim_{n \rightarrow \infty} s_n := \sum_{k=0}^{\infty}a_k
\end{equation*}
$s$ heißt die Summe oder der Wert der Reihe.\\
Falls $s_n$ reell und geht gegen $+\infty$, so schreiben wir:
\begin{equation*}
	\sum_{k=0}^\infty a_k = \infty
\end{equation*}
Gleiches gilt auch für $-\infty$.

\subsection{Notwendige Bedingung für Konvergenz}
Falls die Reihe $\sum_{k=0}^{n}a_k$ konvergent, so gilt: $\lim_{n \rightarrow \infty} a_n = 0$.

\subsection{Konvergenz $\Leftrightarrow$ Beschränktheit bei positiven Gliedern}
Eine reelle Reihe $s_n$ mit positiven Gliedern ist beschränkt gdw. konvergent.

\setcounter{subsection2}{\value{subsection}}
\section*{Vergleichskriterien für Konvergenz}
\addcontentsline{toc}{subsection}{Vergleichskriterien für Konvergenz}
\setcounter{subsection}{\value{subsection2}}
\subsection{Definition Majorante / Minorante}
Sei $s_n$ eine Reihe mit den komplexen Gliedern $a_k$. Eine Reihe $\sum_{k=0}^{n}b_k$ heißt Majorante von $s_n$, wenn $\vert a_k\vert \leq b_k$. Sie heißt Minorante von $s_n$, wenn $b_k \leq \vert{}a_k\vert$.

\subsection{Majorantenkriterium / Minorantenkriterium}
Sei $(s_n)$ eine Reihe mit den komplexen Gliedern $a_k$ und einer konvergenten Majorante mit den Gliedern $b_n$. Dann ist $(s_n)$ konvergent.
Es gilt:
\begin{equation*}
	\left\vert\sum_{k=0}^{\infty}a_k\right\vert \leq \sum_{k=0}^\infty \vert a_k\vert \leq \sum_{k=0}^\infty b_k
\end{equation*}
Falls eine Reihe $(s_n)$ hingegen eine divergente Minorante hat, so ist auch $(s_n)$ divergent.


\subsection{Divergenz}
Eine Reihe heißt divergent, wenn sie nicht konvergent ist.

\subsection{Quotientenkriterium}
Eine Reihe $\sum_{k=0}^\infty a_k$ konvergiert (absolut), falls
\begin{equation*}
	\exists q \in \mathbb{R} \text{ mit } q < 1: \;\; \exists n_0 \geq 0: \;\; \forall k \geq n_0: \;\; \frac{\vert a_{k+1}\vert}{\vert a_k\vert} \leq q
\end{equation*}
Wobei $q$ fest sein muss, also $\frac{\vert a_{k+1}\vert}{\vert a_k\vert} < 1$ reicht nicht aus!

\setcounter{subsection2}{\value{subsection}}
\section*{Alternierende Reihen}
\addcontentsline{toc}{subsection}{Alternierende Reihen}
\setcounter{subsection}{\value{subsection2}}
Eine Reihe mit den Gliedern $a_k$ heißt alternierend, falls die Glieder abwechselnd verschiedene Vorzeichen haben. Wir schreiben: $a_0 - a_1 + a_2 - a_3 + a_4 + ...$.\\
Wir nehmen an: $a_0 > 0$.

\subsection{Leibnitzkriterium}
Sei $(a_n)_{n\geq0}$ eine monoton fallende Folge in $\mathbb{R}$ mit $a \xrightarrow[n \rightarrow \infty]{} 0$ und sei $S_n = \sum\limits_{k=0}^n(-1)^ka_k$.\\
Dann konvergiert die Reihe $\sum\limits_{k=0}^\infty(-1)^ka_k$.\\
Es gilt $\forall n \in \mathbb{N}$:
\begin{equation*}
	\left\vert \sum_{k=0}^{\infty}(-1)^ka_k-S_n\right\vert = \left\vert \sum_{k=n+1}^{\infty}(-1)^ka_k\right\vert \leq a_{n+1}
\end{equation*}

\subsection{Absolute Konvergenz}
$\sum\limits_{k=0}^na_k$ mit $a_k \in \mathbb{C}$ heißt absolut konvergent, falls $\sum\limits_{k=0}^n\vert a_k\vert$ konvergent.\\
Eine Reihe, die konvergiert, aber nicht absolut konvergiert, heißt bedingt konvergent.

\subsection{Umordnungssatz}
Sei $a_k$ eine Folge mit $a_k \in \mathbb{C}$.
\begin{equation*}
	\sum_{k=1}^na_k \text{ konvergiert gdw. } \forall\sigma: \mathbb{N} \rightarrow \mathbb{N} \;:\; \sum_{k=1}^\infty a_{\sigma(k)} = \sum_{k=1}^\infty a_k
\end{equation*}

\subsubsection*{Beliebigkeit der Ergebnisse bei Änderung der Summationsreihenfolge}
Sei $\sum\limits_{n=1}^{\infty}a_n$ eine bedingt konvergente Reihe.\\
Dann existiert je eine Permutation $\sigma$, sodass
\begin{itemize}
	\item $\forall x \in \mathbb{R} \;:\; \sum\limits_{n=1}^{\infty}a_{\sigma(n)} = x$
	\item $\sum\limits_{n=1}^{\infty}a_{\sigma(n)} = -\infty$ oder $\sum\limits_{n=1}^{\infty}a_{\sigma(n)} = \infty$
	\item $\limsup \sum\limits_{k=1}^{n}a_{\sigma(k)} \neq \liminf \sum\limits_{k=1}^{n}a_{\sigma(k)}$
\end{itemize}
%und sei $x \in \mathbb{R}$. Dann $\exists \sigma$ Permutation, sodass $\sum\limits_{n=1}^{\infty}a_{\sigma(n)} = x$.
%
%\noindent Es gibt auch eine Permutation $\sigma$, sodass $\sum\limits_{n=1}^{\infty}a_{\sigma(n)}$ gegen $+\infty$ konvergiert, eine, sodass es gegen $-\infty$ konvergiert und eine, wo $\limsup \sum\limits_{k=1}^{n}a_{\sigma(k)} \neq \liminf \sum\limits_{k=1}^{n}a_{\sigma(k)}$.

\subsection{Doppelreihensatz}
Seien $a_{i,j} \in \mathbb{C}$ mit $i,j \in \mathbb{N}$. Falls $\sum\limits_{i=0}^{\infty}\sum\limits_{j=0}^\infty\vert a_{i,j}\vert < \infty$, dann konvergiert $\sum\limits_{i=0}^{\infty}\sum\limits_{j=0}^\infty a_{i,j}$ und
\begin{align*}
\sum\limits_{i=0}^{\infty}\sum\limits_{j=0}^\infty a_{i,j} &= \sum\limits_{j=0}^{\infty}\sum\limits_{i=0}^\infty a_{i,j} \\
&= \sum\limits_{n=0}^{\infty}\sum\limits_{k=0}^\infty a_{k,(n-k)}
\end{align*}

\noindent Es folgt:

Seien $a_n, b_n \in \mathbb{R}$ reelle Folgen. Dann gilt:
\begin{align*}
\sum_{k=0}^{\infty}a_k \sum_{j=0}^{\infty}b_j &= \sum_{k=0}^{\infty}\sum_{j=0}^\infty a_kb_j \\
&= \sum_{m=0}^{\infty}\sum_{k,j,k+j = m}^{\infty} a_kb_j \\
&= \sum_{m=0}^{\infty} c_m
\end{align*}
wobei $c_m = \sum\limits_{k=0}^ma_kb_{m-k}$ die Definition des Cauchy-Produkts.

\subsection{Exponentialfunktion}
\begin{equation*}
	\exp(z) = \sum_{k=0}^\infty\frac{z^k}{k\!} \hspace{1cm}\forall z \in \mathbb{C}
\end{equation*}
Dabei gilt: $\exp(z) = e^z$ mit $e$ die Eulersche Zahl $\forall z \in \mathbb{C}$.

\subsection{Rechenregeln der Exponentialfunktion}
Seien $z \in \mathbb{C}$, $x \in \mathbb{R}$ und $n \in \mathbb{N}$. Dann gelten:
\begin{enumerate}
	\item $\exp(-z) = \frac{1}{\exp(z)}$
	\item $\exp(z) \neq 0+0i$
	\item $\exp(x) > 0$
	\item $\overline{\exp(z)} = \exp \overline{z}$
	\item $\exp(x)$ ist monoton wachsend
	\item $\vert \exp(z)\vert \leq \exp (\vert z\vert)$
\end{enumerate}


\subsection*{Überblick über alle Kriterien}
Kriterien:
\begin{itemize}
	\item Majorantenkriterium
	\item Minoratenkriterium
	\item Quotientenkriterium
	\item Wurzelkriterium*
	\item Nullfolgenkriterium*
	\item Leibnizkriterium
\end{itemize}
* Wurden nicht angesprochen

\newpage
\section{Grenzwerte von Funktionen und Stetigkeit}
\noindent Sämtliche Sätze gelten auch auf $\mathbb{C}$, auch wenn sie nur auf $\mathbb{R}$ angegeben sind.
\subsection{Definition Isolierter Punkt}
Sei $D \subseteq \mathbb{R}$ und $x_0 \in D$. $x_0$ heißt isoliert in $D$ wenn:
\begin{equation*}
\nexists (a_n)_{n \in \mathbb{N}} \text{ mit } a_n \in D \backslash \{x_0\}: \;\; \lim a_n = x_0
\end{equation*}

\subsection{Grenzwert einer Funktion}
Sei $D \subseteq \mathbb{R}$, $f: D \rightarrow \mathbb{R}$ und $x_0 \in D$.\\
$a$ heißt Grenzwert von $f$ in $x_0$ (Schreibweise: $a = \lim\limits_{x \rightarrow x_0} f(x)$), falls 
\begin{enumerate}[label=\alph*.]
	\item $x_0$ nicht isoliert und
	\item $\forall (a_n)_{n \in \mathbb{N}}$ mit $a_n \in D$ und $\lim\limits_{n \rightarrow \infty} a_n = x_0$ gilt: $\lim\limits_{n \rightarrow \infty} f(a_n) = a$
\end{enumerate}

\subsection{Definition Stetigkeit}
Sei $D \subseteq \mathbb{R}$, $f: D \rightarrow \mathbb{R}$ und $x_0 \in D$.\\
$f$ heißt stetig in $x_0$, falls $x_0$ isoliert oder $\lim\limits_{x \rightarrow x_0} f(x) = f(x_0)$.\\
$f$ heißt stetig in $D$, falls $\forall x \in D$ $f$ stetig in $x$.

\subsection{Alternative Definition für Grenzwert}
Sei $D \subseteq \mathbb{R}$, $f: D \rightarrow \mathbb{R}$ und $x_0 \in D$ nicht isoliert.
\begin{equation*}
	\lim\limits_{x \rightarrow x_0} f(x) = a \text{ gdw. } \forall \varepsilon > 0: \;\; \exists \delta > 0: \;\; \forall x \in (x_0 - \delta, x_0 + \delta) \cap (D \backslash \{x_0\}): \;\; \vert f(x) - a\vert < \varepsilon
\end{equation*}

\subsection{Rechenregeln beim Grenzwert}
Sei $D \subseteq \mathbb{R}$ und seien $f, g: D \rightarrow \mathbb{R}$. Dann gilt:
$\forall x \in D$ mit $f$ und $g$ stetig in $x$: $f + g, f-g, f\cdot g$ stetig in x. Zudem $\frac{f}{g}$ stetig in $x$, falls $g(x) \neq 0$.\\\\
Damit sind endliche Polynome (und Brüche daraus) auf ihrem Definitionsbereich stetig.

\subsection{Stetigkeit von $\exp$}
Die Exponentialfunktion $\exp$ ist stetig auf $\mathbb{C}$.

\subsection{Komposition stetiger Funktionen}
Seien $f: D_f \rightarrow R$, $g : D_g \rightarrow \mathbb{R}$ und $f(D_f) \subseteq D_g$ \footnote{$f(A) := \left\{f(x) \;\vert\; x \in A\right\}$}.\\
$\forall x \in D_f$: Wenn $f$ stetig in $x$ und $g$ stetig in $f(x)$, so ist auch $g \circ f$ stetig in $x$.

\subsection{Definition linksseitiger / rechtsseitiger Grenzwert / Stetigkeit}
Seien $D \subseteq \mathbb{R}$, $f : D \rightarrow \mathbb{R}$ und $a \in D$ nicht isoliert.\\
$c \in \mathbb{R}$ heißt linksseitiger Grenzwert von $f$ im Punkt $a$ (Schreibweise: $\lim\limits_{x \rightarrow a^-} f(x) = c$)\footnote{alternativ auch $\lim\limits_{x \nearrow a} f(x) = c$ für linksseitigen und $\lim\limits_{x \searrow a} f(x) = c$ für rechtsseitigen Grenzwert}, falls $\forall (x_n)_{n \in \mathbb{N}}$ mit $x_n \in D$, $\forall n \in \mathbb{N}: x_n < a$ und $x_n \rightarrow c$ gilt: $\lim_{n \rightarrow \infty} f(x_n) = c$.\\
$f$ heißt linksseitig stetig in $a$, falls $\lim\limits_{x \rightarrow a^-} f(x) = f(a)$.\\\\
Analog ist der rechtsseitige Grenzwert definiert: $\lim\limits_{x \rightarrow a^+} f(x)$


\subsection*{Fixpunktiteration}
\addcontentsline{toc}{subsection}{Fixpunktiteration}
Sei $f : \mathbb{R} \rightarrow \mathbb{R}$. $x \in \mathbb{R}$ heißt Fixpunkt von $f$, falls $f(x) = x$.\\
Iterativer Lösungsansatz: $x_{n+1} = f(x_n)$. Dann gilt: Falls $f$ stetig in $D$, $f(D) \subseteq D$ und $x_0 \in D$, so ist $\lim_{n \rightarrow \infty}x_n$ ein Fixpunkt.

\newpage
\section{Komplexe Zahlen und trigonometrische Funktionen}
Wir rechnen in Bogenmaßen. Der Einheitskreis ist definiert als $\big\{z \;\big\vert\; \vert z\vert = 1\big\}$.

\subsection{Definition Sinus / Cosinus}
\begin{align*}
	\sin x &:= \frac{\exp(ix) - \exp(-ix)}{2i} \;=\; \sum_{n=0}^{\infty}\frac{(-1)^nx^{2n+1}}{(2n+1)!}\\
	\cos x &:= \frac{\exp(ix) + \exp(-ix)}{2} \;=\; \sum_{n=0}^{\infty}\frac{(-1)^nx^{2n}}{(2n)!}
\end{align*}
Es folgt: $\sin 0 = 0$ und $\cos 0 = 1$\\
Diese Definition ist im reellen Raum äquivalent zur geometrischen Definition am rechtwinkligen Dreieck.

\subsection{Sinus / Cosinus periodisch, Definition Pi}
$\sin$ und $\cos$ sind periodisch, d.h. $\exists \pi \in \mathbb{R}: \forall z \in \mathbb{C} : \sin z + 2 \pi = \sin z, \cos z + 2\pi = \cos z$.\\
$\pi \approx 3,14159265412$.

\subsection*{Rechenregeln für Sinus und Cosinus}
\addcontentsline{toc}{subsection}{Rechenregeln für Sinus und Cosinus}
\begin{align*}
	\sin -x                              & = -\sin x  &  & \text{Punktsymmetrie}  &  &  &  &  \\
	\cos -x                              & = \cos x   &  & \text{Achsensymmetrie} &  \\
	\sin \left(x + \sfrac{\pi}{2}\right) & = \cos x   &  & \text{Verschiebung}    &  \\
	\cos \left(x + \sfrac{\pi}{2}\right) & = - \sin x &
\end{align*}
Zudem folgt aus dem Satz des Pythagoras:
\begin{equation*}
	\sin^2 x + \cos^2 x = 1
\end{equation*}

\subsection*{Rechenregeln für komplexe Exponentialfunktion}
\addcontentsline{toc}{subsection}{Rechenregeln für komplexe Exponentialfunktion}
Sei $x \in \mathbb{R}$ und $z = \exp ix = e^{ix}$.
\begin{align*}
	\vert e^{ix}\vert & = e^{ix}\overline{e^{ix}} \\
	                  & = e^{ix}e^{\overline{ix}} \\
	                  & = e^{ix}e^{-ix}           \\
	                  & = e^0                     \\
	                  & = 1
\end{align*}
Also ist $e^{ix}$ der Einheitskreis auf der komplexen Ebene:
\begin{equation*}
	e^{ix} = \cos x + i \sin x
\end{equation*}
Insbesondere gilt also:
\begin{align*}
	e^{i\frac{\pi}{2}} &= i & e^{i\pi} = -1
\end{align*}
Folgerung:
\begin{equation*}
	e^{-ix} = \cos x - i \cdot \sin x
\end{equation*}

\subsubsection*{Definition Tangens / Kotangens}
\addcontentsline{toc}{subsection}{Definition Tangens / Kotangens}
\begin{align*}
	\tan x & := \frac{\sin x}{\cos x}                                         \\
	\cot x & := \frac{\cos x}{\sin x} \underset{\sin x \neq 0}{=} \tan^{-1} x
\end{align*}
Hinweis: $\tan x$ und $\cot x$ sind periodisch (Periode $\pi$). $\tan \frac{\pi}{2}$ und $\cot 0$ sind nicht definiert.

\subsection{Konvergenz bei Multiplikation}
Seien $d \in \mathbb{N}$, $D \subseteq \mathbb{R}^d$, $a \in D$ und $f,g : D \rightarrow \mathbb{R}$ und $g$ auf $D$ beschränkt sowie \mbox{$\lim\limits_{x \rightarrow a} f(x) = 0$}.
Dann folgt: $\lim\limits_{x \rightarrow a} f(x) \cdot g(x) = 0$.

\subsection*{Polarkoordinaten}
\addcontentsline{toc}{subsection}{Polarkoordinaten}
Sei $z \in \mathbb{C} \backslash \{0\}$. Dann gibt es $r, \phi \in \mathbb{R}$, sodass $r = \vert z\vert$ und $e^{i\phi} = \frac{z}{r}$ (da $\vert\frac{z}{r}\vert = 1$).\\
$r$ und $\phi$ heißen Polarkoordinaten von $z$: $r$ ist der Abstand von $z$ zum Nullpunkt, $\phi$ der Winkel zur reellen Achse. Die Abbildung Polarkoordinaten $\leftrightarrow$ Real- und Imaginärteil ist bijektiv.\\
Die Multiplikation in Polarkoordinaten\footnote{Ich (Josef) schreibe: $r \angle \phi$ für $r \cdot e^{i\phi}$}: $z_1 = r_1 \angle \phi_1$, $z_2 = r_2 \angle \phi_2$. Es folgt: $z_1 \cdot z_2 = (r_1\cdot r_2) \angle (\phi_1 + \phi_2)$

\subsection*{Nützliche Rechenregeln}
\addcontentsline{toc}{subsection}{Nützliche Rechenregeln}
\begin{itemize}
	\item $\sin 2\alpha = 2\sin \alpha \cos \alpha$
	\item $\cos 2 \alpha = \cos^2\alpha - \sin^2\alpha = 2\cos^2 \alpha - 1$
	\item $\begin{aligned}[t]
		\sin(\alpha + \beta) &= \sin \alpha \cos \beta + \cos\alpha\sin\beta & \sin(\alpha - \beta) &= \sin \alpha \cos \beta - \cos\alpha\sin\beta \\
		\cos(\alpha + \beta) &= \cos \alpha \cos \beta - \sin\alpha\sin\beta & \cos(\alpha - \beta) &= \cos \alpha \cos \beta + \sin\alpha\sin\beta
	\end{aligned}$
\end{itemize}

\newpage
\section{Konsequenzen der Stetigkeit}
\subsection{Unvollständigkeit des Ergebnisraums einer beliebigen Funktion}
Sei $f : [a,b] \rightarrow \mathbb{R}$.\\
Es muss $\forall y \in [f(a), f(b)]$ \underline{nicht unbedingt} auch ein $x$ mit $f(x) = y$ geben.

\subsection{Zwischenwertsatz}
$f : [a,b] \rightarrow \mathbb{R}$.\\
Wenn $f$ stetig, so $\forall y\in[f(a), f(b)]: \exists x : f(x) = y$.

\subsection{Maximum / Minimum}
Sei $D \subseteq \mathbb{R}$, $f : D \rightarrow \mathbb{R}$.\\
$x \in \mathbb{R}$ heißt Maximum von $f$, wenn $\forall z \in D: f(x) \geq f(z)$.\\
$x \in \mathbb{R}$ heißt Minimum von $f$, wenn $\forall z \in D: f(x) \leq f(z)$.\\\\
Hinweis: Maximum und Minimum sind nicht unbedingt eindeutig. Nicht jede Funktion hat ein Maximum und/oder Minimum.

\stepcounter{subsection}
\subsection{Maximum/Minimum einer stetigen Funktion}
Jede stetige Funktion $f: [a, b] \rightarrow \mathbb{R}$ hat ein Maximum und ein Minimum.

\subsection{Definition Konvergenz \& Stetigkeit im mehrdimensionalen Raum}
Sei $d \in \mathbb{N}$, $(x_n)_n \in \mathbb{N}$ eine Folge in $\mathbb{R}^d$ und $x \in \mathbb{R}^d$.\\
$x_n$ konvergiert gegen $x$ falls $\lim\limits_{n \rightarrow \infty} \vert\vert x_n - x\vert\vert = 0$. (Schreibweise: $\lim\limits_{n \rightarrow \infty} x_n = x$ 
oder \mbox{$x_n \xrightarrow[n \rightarrow \infty]{} x$})\\
Alternativ komponentenweise: $\forall i \in \{1,...,d\}: \lim\limits_{n \rightarrow \infty} {x_n}_i = x_i$.\\\\
Sei $d \in \mathbb{N}$, $D \subseteq \mathbb{R}^d$, $x \in D$ und $f : D \rightarrow \mathbb{R}^m$ mit $m \in \mathbb{N}$.\\
$f$ heißt stetig in $x \in D$, falls $\forall (x_n)$ in D mit $x_n \xrightarrow[n \rightarrow \infty]{} x$ gilt: $\lim\limits_{n \rightarrow \infty} = f(x)$.\\
$f$ heißt stetig, falls $\forall x \in D$ gilt: $f$ stetig in $x$.

\subsection*{mehrdimensionale Offenheit / (Ab-)Geschlossenheit}
\addcontentsline{toc}{subsection}{mehrdimensionale Offenheit / Geschlossenheit}
Eine Menge $A \subseteq \mathbb{R}^n$ heißt offen, wenn $\forall x_0 \in A: \exists \varepsilon > 0 : \{x \mid \Vert x-x_0\Vert < \varepsilon\} \subseteq A$.\\
$A$ heißt geschlossen, wenn $\mathbb{R}^n \backslash A$ offen.\\
$A$ ist (ab-)geschlossen gdw. $\forall (a_n)$ mit $a_n \in A$ und $\lim\limits_{n \rightarrow \infty} a_n = a$ gilt: $a \in A$\\\\
Anmerkung: Nur $\mathbb{R}^n$ und $\emptyset$ sind zugleich offen und geschlossen.

\subsection{Definition Kompaktheit}
Sei $n \in \mathbb{N}$ und $A \subseteq \mathbb{C}^n$.\\
$A$ heißt kompakt, falls $\forall (a_n)$ mit $a_n \in A$ gilt: $\exists (a_{n_k})$ Teilfolge von $(a_n)$, sodass \mbox{$\lim\limits_{k \rightarrow \infty} a_{n_k}\in{}A$}.

\subsection{stetig von Kompakter Menge abbilden $\rightarrow$ Maxi- und Minimum}
Sei $A$ kompakt und $f : A \rightarrow \mathbb{R}$.\\
Wenn $A$ stetig, so hat $f$ sowohl ein Maximum als auch ein Minimum.

\subsection{mehrdimensionale Beschränktheit}
$M \subseteq \mathbb{R}^d$ heißt beschränkt, wenn $\exists K \in \mathbb{R} : \forall x \in M : \Vert x\Vert < K$.\\
Eine Folge $(x_n) \in \mathbb{R}^n$ heißt beschränkt, wenn die Menge aller ihrer Glieder beschränkt ist. Wir schreiben die Glieder $\forall n: x_n = (x_{n,1},...,x_{n,d})$ mit $x_{n,k} \in \mathbb{R}$ und $1 \leq k \leq d$.

\subsection{komponentenweise Konvergenz}
Eine Folge $(x_n)$ in $\mathbb{R}^d$ konvergiert gdw. alle ihre Komponenten konvergieren:
\begin{align*}
	\lim_{n \rightarrow \infty} x_n &= x &  & \text{ gdw. } &  &\forall k \in \{1,...,d\}: \lim_{n \rightarrow \infty} x_{n,k} = x_k
\end{align*}

\subsection{Kompakt $\leftrightarrow$ abgeschlossen \& beschränkt}
$A \subseteq \mathbb{R}^n$ ist kompakt gdw. $A$ ist abgeschlossen und beschränkt.

\subsection{Kompaktheit steter Bilder kompakter Mengen}
Sei $D \subseteq \mathbb{R}^d$ kompakt und $f : \mathbb{R}^d \rightarrow \mathbb{R}^m$ stetig.\\
Dann ist $f(D)$ kompakt.\footnote{$f(D) = \left\{f(x) \;\middle|\; x \in D\right\}$}

\subsection{Definition Umkehrabbildung \& -funktion}
Sei $f: A \rightarrow B$ bijektiv\footnote{bijektiv heißt injektiv und surjektiv:\\Injektiv: auf jedes Element wird max. 1x abgebildet\\Surjektiv: auf jedes Element wird min. 1x abgebildet}.\\
$\exists f^{-1} : B \rightarrow A$ Umkehrabbildung, sodass $\forall y \in B: f(f^{-1}(y)) = y$.\\\\
Wenn $A, B \subseteq \mathbb{R}^d$, so heißt $f^{-1}$ Umkehrfunktion.\\\\
Hinweis: $f^{-1}$ ist bijektiv. Man erhält den Graphen von $f^{-1}$ durch Spiegelung des Graphen von $f$ an der Geraden $y=x$, wenn $A, B \subseteq \mathbb{R}$.

\subsection{Umkehrfunktionen von stetigen, streng monoton wachsenden Funktionen}
Sei $I \subseteq \mathbb{R}$ ein Intervall, $f : I \rightarrow \mathbb{R}$ stetig und streng monoton wachsen.\\
$f$ ist bijektiv, $f^{-1}$ ist auch stetig und streng monoton wachsend.

\subsection{Definition natürlicher Logarithmus}
\begin{equation*}
	\ln := \exp^{-1} \text{ mit } \exp: \mathbb{R} \rightarrow (0, \infty)
\end{equation*}
heißt (natürlicher) Logarithmus.

\subsection*{Rechenregeln des Logarithmus}
\addcontentsline{toc}{subsection}{Rechenregeln des Logarithmus}
Seien $x,y \in \mathbb{R}, k \in \mathbb{Z}$.
\begin{align*}
	\ln(1) &= 0 \\
	\ln(e) &= 1 \\
	\ln(x) + \ln(y) &= \ln(x \cdot y)\\
	\ln(x^k) &= k \cdot \ln(x)\;\;\text{ falls }x > 0
\end{align*}

\subsection{Definition Potenzieren auf $\mathbb{R}$, Wurzelrechnung}
Sei $x,a \in \mathbb{R}$ mit $x > 0$.
\begin{equation*}
	x^a := \exp(a \cdot \ln(x))
\end{equation*}
Es gilt: $x^{a+b} = \exp\big((a+b) \cdot \ln(x)\big) = \exp\big(a \ln(x)\big)\cdot \exp\big(b \ln(x)\big) = x^ax^b$.\\\\
Wir schreiben: $\sqrt[\uproot{1}n]{x} := x^\frac{1}{n}$.\\
Hinweis: Potenzfunktionen sind im Allgemeinen nicht bijektiv, Wurzelfunktion sind daher oft keine vollständige Umkehrfunktion!

\subsection*{Definition Logarithmus zu Basen}
\addcontentsline{toc}{subsection}{Definition Logarithmus zu Basen}
Sei $x, b \in \mathbb{R}$ mit $b > 1$.
\begin{equation*}
	\log_b(x) := \frac{\ln(x)}{\ln(n)}
\end{equation*}
ist die Umkehrfunktion von $b^x$ und heißt Logarithmus zur Basis $b$.

\subsection*{Umkehrfunktionen der trigonometrischen Funktionen}
\addcontentsline{toc}{subsection}{Umkehrfunktionen der trigonometrischen Funktionen}
Die trigonometrischen Funktionen sind im allgemein nur in einem gewissen Bereich umkehrbar, da sie im allgemeinen nicht bijektiv sind.
\subsubsection*{Tangens}
Die Umkehrfunktion von $\tan x$ für $x \in \left(-\frac{\pi}{2},\frac{\pi}{2}\right)$:
\begin{equation*}
	\arctan : \mathbb{R} \rightarrow \left(-\frac{\pi}{2},\frac{\pi}{2}\right)
\end{equation*}
heißt Arcustangens.
\subsubsection*{Sinus}
Die Umkehrfunktion von $\sin x$ für $x \in \left(-\frac{\pi}{2},\frac{\pi}{2}\right)$:
\begin{equation*}
\arcsin : \left[-1,1\right] \rightarrow \left(-\frac{\pi}{2},\frac{\pi}{2}\right)
\end{equation*}
heißt Arcussinus.
\subsubsection*{Cosinus}
Die Umkehrfunktion von $\cos x$ für $x \in \left(0,\pi\right)$:
\begin{equation*}
\arccos : \left[-1,1\right] \rightarrow \left(0,\pi\right)
\end{equation*}
heißt Arcuscosinus.\\

\newpage
\section{Differentiation}
\subsection{Definition $\mathcal{O}$ und $o$}
Seien $f,g : \mathbb{C} \rightarrow \mathbb{C}$ und $a \in \mathbb{R} \cup \left\{-\infty, \infty\right\}$.\\
Wir schreiben: $f(x) = \mathcal{O}(g(x))$ für $x \rightarrow a$ falls:
\begin{equation*}
	\exists c > 0 \;:\; \forall (x_n) \in \mathbb{C} \text{ mit } x_n \rightarrow a \;:\; \left|f(x_n)\right| \leq c \cdot \left|g(x_n)\right|
\end{equation*}
für \underline{alle bis auf endlich viele $n$}.\\
Wir schreiben: $f(x) = o(g(x))$ für $x \rightarrow a$ falls
\begin{equation*}
	\lim_{x\rightarrow a}\frac{f(x)}{g(x)} = 0
\end{equation*}
Sprich: \glqq{}$f$ ist gegenüber $g$ asymptotisch vernachlässigbar für $x \rightarrow a$.\grqq

\stepcounter{subsection}
\subsection{Definition innerer Punkt}
Sei $D \subseteq \mathbb{R}$. $x_0 \in D$ heißt innerer Punkt von $D$, falls:
\begin{equation*}
	\exists \varepsilon > 0 \;:\; (x_0 - \varepsilon, x_0 + \varepsilon) \subseteq D
\end{equation*}
Hinweis: $D$ ist offen falls alle $x_0 \in D$ innere Punkte von $D$ sind.

\subsection{Definition Differenzierbarkeit einer Stelle \& Ableitung}
Sei $D \subseteq \mathbb{R}, f: D \rightarrow \mathbb{R}$ und $x_0$ innerer Punkt in $D$.\\
$f$ heißt Differenzierbar in $x_0$, wenn:
\begin{equation*}
	\exists f'(x_0) \;:\; x \in D \text{ mit } \lim_{n \rightarrow \infty}x = x_0 \;:\; f(x) = f(x_0) + f'(x_0)(x-x_0)+o(x-x_0)
\end{equation*}
$f'(x)$ heißt die Ableitung von $f$ und beschreibt die Steigung der Tangenten an $f(x)$.\\
Ausgeschrieben lautet die Definition:
\begin{equation*}
	f'(x_0) = \lim_{x\rightarrow x_0} \frac{f(x) - f(x_0)}{x - x_0} = \lim_{h\rightarrow 0} \frac{f(x_0 + h) - f(x_0)}{h}
\end{equation*}

\subsection{Definition Differenzierbarkeit einer Funktion}
Sei $D \subseteq \mathbb{R}$ offen, $f : D \rightarrow \mathbb{R}$.\\
$f$ heißt differenzierbar in $D$, falls $f$ differenzierbar in $x$, $\forall x \in D$.

\addtocounter{subsection}{2}
\subsection{Differenzierbarkeit $\rightarrow$ Stetigkeit}
Sei $D \subseteq \mathbb{R}$, $x_0 \in D$ und $f : D \rightarrow \mathbb{R}$ differenzierbar in $x_0$.\\
Dann ist $f$ stetig in $x_0$.\\\\
\underline{Wichtig:} Die Umkehrung gilt nicht, d.h. aus Stetigkeit lässt nicht auf Differenzierbarkeit schließen! Bsp: $f(x) = \vert x\vert$ an $x = 0$

\stepcounter{subsection}
\subsection{Definition rechts- \& linksseitige Ableitung}
Sei $D \subseteq \mathbb{R}, f: D \rightarrow \mathbb{R}$ und $x_0$ innerer Punkt in $D$.\\
\begin{equation*}
	f'_+(x_0) = \lim_{\stackrel{h\rightarrow 0}{h > 0}}\frac{f(x_0 + h) - f(x_0)}{h}
\end{equation*}
heißt rechtsseitige Ableitung von $f$ in $x_0$.
\begin{equation*}
f'_-(x_0) = \lim_{\stackrel{h\rightarrow 0}{h < 0}}\frac{f(x_0 + h) - f(x_0)}{h}
\end{equation*}
heißt linksseitige Ableitung von $f$ in $x_0$.

\subsection*{Notation der Ableitung}
Die Schreibweise $f'(x_0)$ für die Ableitung von $f$ heißt Lagrange-Notation. Bekannt ist auch die Leibniz-Notation $\frac{df}{dx}(x_0)$.

\subsection*{Definition Differentialoperator}
Sei $I$ ein offenes Intervall.\\
$D: f \mapsto f'$ ordnet jeder in $I$ differenzierbaren Funktion $f$ ihre Ableitung $f'$ zu.\\\\
$D$ ist linear, d.h. $(a f)'(x) = a \cdot f'(x)$ und $(f + g)'(x) = f'(x) + g'(x)$.

\subsection*{Ableitung der trigonometrischen Funktionen}
\begin{align*}
	\sin' &= \cos\\
	\cos' &= -\sin
\end{align*}

\addtocounter{subsection}{4}
\subsection{Produktregel}
Sei $x \in \mathbb{R}$ und seien $f, g: \mathbb{R} \rightarrow \mathbb{R}$ differenzierbar in $x$.\\
\begin{equation*}
	(f \cdot g)'(x) = f'(x)\cdot g(x) + f(x)\cdot g'(x)
\end{equation*}
Daraus folgt auch: $(x^n)' = n \cdot x^{n-1}$.

\stepcounter{subsection}
\subsection{Quotientenregel}
Sei $x \in \mathbb{R}$ und seien $f, g: \mathbb{R} \rightarrow \mathbb{R}$ differenzierbar in $x$, wobei $g(x) \neq 0$.\\
\begin{equation*}
	\left(\frac{f}{g}\right)'(x) = \frac{f'(x)g(x) - f(x)g'(x)}{g(x)^2}
\end{equation*}

\stepcounter{subsection}
\subsection{Kettenregel}
Seien $I, J \subseteq \mathbb{R}$, $f: I \rightarrow \mathbb{R}$ mit $f(I) \subseteq J$, $g: J \rightarrow \mathbb{R}$, $x_0 \in I$ innerer Punkt von $I$ und $f(x_0)$ innerer Punkt von $J$.\\
$g \circ f$ ist in $x_0$ differenzierbar, wenn $f$ in $x_0$ und $g$ in $g(x_0)$ differenzierbar. Es gilt:
\begin{equation*}
	(g \circ f)'(x) = g'(f(x_0)) \cdot f'(x_0)
\end{equation*}

\subsection{Ableitung der Potenzierung}
Sei $a > 0$ und $h(x) = a^x$ mit $D = \mathbb{R}$. Dann gilt:
\begin{equation*}
	\frac{da^x}{x}=a^x\cdot \ln(a)
\end{equation*}

\addtocounter{subsection}{2}
\subsection{Umkehrregel}
Sei $f: [a, b] \rightarrow [c, d]$ bijektiv und sei $y \in [a, b]$ in $f$ differenzierbar (mit $f'(y) \neq 0$).\\
Es folgt: $f^{-1}$ ist in $x = f(y)$ differenzierbar und:
\begin{equation*}
	\left(f^{-1}\right)'(y) = \frac{1}{f'\left(f^{-1}(y)\right)}
\end{equation*}

\addtocounter{subsection}{3}
\subsection{Definition Konvergenz von Funktionsfolgen}
Sei $I \subseteq R^n$ und $f_k$ mit $\forall k \in \mathbb{N}: f_k: I \rightarrow \mathbb{R}$. Wir definieren den Grenzwert $f : I \rightarrow \mathbb{R}$:
\begin{align*}
	&\textbf{punktweise Konvergenz:} & &\forall \varepsilon > 0: \forall x \in I: \exists N \in \mathbb{N} : \forall n \geq N: \vert f_n(x)-f(x)\vert < \varepsilon\\
	&\textbf{gleichmäßige Konvergenz:} & &\forall \varepsilon > 0: \exists N \in \mathbb{N} : \forall x \in I: \forall n \geq N: \vert f_n(x)-f(x)\vert < \varepsilon
\end{align*}
Gleichmäßige Konvergenz impliziert damit punktweise Konvergenz. (Aber nicht anders herum, vgl. $f_k(x) = x \geq k \texttt{ ? } 1 \texttt{ : } 0$ mit $f(x) = 0$) \\\\
Diese Konvergenzbegriffe lassen sich auch auf Reihen anwenden, dann ist $f(x)$ der Grenzwert der Reihe $\sum\limits_{k=0}^\infty f_k(x)$.

\subsection{Differenzierbarkeit \& Ableitung von Funktionsfolgen}
Sei $I \subseteq R$ offen und $f_k$ eine Funktionsfolge von $I \rightarrow \mathbb{R}$. \\
Wenn $f_k$ punktweise und $f'_k$ gleichmäßig konvergiert, so ist $f'$ differenzierbar in $I$ und $f'(x) = \lim\limits_{k\rightarrow \infty}f'_k$.\\\\
Für Reihen: Wenn $\sum\limits_{k=0}^\infty f_k$ punktweise und $\sum\limits_{k=0}^\infty f'_k$ gleichmäßig konvergieren, so ist $f'(x) = \sum\limits_{k=0}^\infty f'_k(x)$.

\newpage
\section{Anwendungen der Ableitung}
\subsection{Definition Extrema einer Funktion}
Sei $I = [a,b]$ und $f:I\rightarrow\mathbb{R}$.\\
\minisec{globale Extrema}
Sofern existent, heißen das Maximum und Minimum von $f$ auf $I$ globale Extrema.
\minisec{lokale Extrema}
$f$ hat in $x_0$ ein lokales Maximum, falls ein $\varepsilon > 0$ existiert, sodass $\forall x \in (x_0 - \varepsilon, x_0 + \varepsilon)\cap I$ gilt: $f(x) \leq f(x_0)$.\\
$f$ hat in $x_0$ ein strikt lokales Maximum, falls ein $\varepsilon > 0$ existiert, sodass $\forall x \in (x_0 - \varepsilon, x_0 + \varepsilon)\cap I$ mit $x \neq x_0$ gilt: $f(x) < f(x_0)$.\\\\
Analog das (strikt) lokale Minimum.\\\\
Jedes globale Extremum ist auch ein entsprechend lokales Extremum.

\subsection{Ableitung an Extrema}
Sei $f:[a,b] \rightarrow\mathbb{R}$ in $(a,b)$ differenzierbar und $x_0 \in (a,b)$ sei lokales Extremum in $f$.\\
Dann gilt: $f'(x_0) = 0$.\\\\
Hinweis: Aus $f'(x) = 0$ folgt nicht sofort ein lokales Extremum in $x$.

\subsection*{Bestimmung von Extrema}
\addcontentsline{toc}{subsection}{Bestimmung von Extrema}
Voraussetzung: Bei allen (inneren) Extrema ist $f(x)$ differenzierbar.
\begin{enumerate}
	\item Bestimme Nullstellen von $f'$ in $(a,b)$
	\item Filtere lokale Extrema heraus
	\item Untersuche Verhalten von $f$ in den Randpunkten
	\item Bestimme größtes/kleinstes Extrema als jeweils globales Maximum/Minimum
\end{enumerate}

\addtocounter{subsection}{3}
\subsection{Satz von Rolle}
Sei $f :[a,b] \rightarrow \mathbb{R}$ stetig und differenzierbar in $(a,b)$.\\
Falls $f(a) = f(b)$, so $\exists z \in (a,b)$ mit $f'(z) = 0$.

\subsection{(erweiterter) Mittelwertsatz}
Seien $f,g : [a,b] \rightarrow \mathbb{R}$ stetig auf $[a,b]$ und differenzierbar in $(a,b)$, wobei $\forall x \in (a,b) : g'(x) \neq 0$.\\
Dann gilt: $g(a) \neq g(b)$ und $\exists z \in (a,b)$ mit $g'(z)\cdot\left(f(b) - f(a)\right) = f'(z)\cdot\left(g(b)-g(a)\right)$ bzw. $\frac{f(b)-f(a)}{g(b)-g(a)} = \frac{f'(z)}{g'(z)}$.\\\\
Spezialfall $g(x) = x$: $\exists z \in (a,b)$ mit $\frac{f(b)-f(a)}{b-a}=f'(z)$.

\subsection*{Zwischenwertsatz für Ableitungen}
Sei $f : [a,b] \rightarrow \mathbb{R}$ differenzierbar.\\
Dann gilt: $\forall c \in [f(a),f(b)] : \exists x \in [a,b] : f'(x) = c$. 
\subsection{Monotonie und Ableitung}
Sei $f : [a,b] \rightarrow \mathbb{R}$ differenzierbar in $(a,b)$.\\
Dann gilt:
\begin{enumerate}
	\item $f'>0$ in $(a,b)$ $\Rightarrow$ $f$ in $(a,b)$ streng monoton wachsend
	\item $f'<0$ in $(a,b)$ $\Rightarrow$ $f$ in $(a,b)$ streng monoton fallend	
	\item $f'\geq0$ in $(a,b)$ $\Rightarrow$ $f$ in $(a,b)$ monoton wachsend
	\item $f'\leq0$ in $(a,b)$ $\Rightarrow$ $f$ in $(a,b)$ monoton fallend
	\item $f' = 0$ in $(a,b)$ $\Rightarrow$ $f$ in $(a,b)$ konstant
\end{enumerate}
Falls $f$ in $a$ oder $b$ stetig, so können diese Randpunkte jeweils hinzugenommen werden.\\

\addtocounter{subsection}{2}
\subsection{Kriterium für Extrema}
Sei $f$ differenzierbar in $(a,b)$ und $x_0 \in (a,b)$ mit $f'(x_0) = 0$.\\
Dann gilt:
\begin{enumerate}
	\item $f' \geq 0$ in $(a,x_0)$ und $f'\leq 0$ in$(x_0,b)$ $\Rightarrow$ $x_0$ lokales Maximum in $f$
	\item $f' \leq 0$ in $(a,x_0)$ und $f'\geq 0$ in$(x_0,b)$ $\Rightarrow$ $x_0$ lokales Minimum in $f$
\end{enumerate}

\subsection{Regel von l'Hospital}
Seien $f,g : (a,b) \rightarrow \mathbb{R}$ differenzierbar (wobei $a = -\infty$ und/oder $b = \infty$ erlaubt) mit $\forall x \in(a,b) : g'(x) \neq 0$ und sei $x_0 \in \{a,b\}$.\\
Falls
\begin{itemize}
	\item $\lim\limits_{x \rightarrow x_0} f(x) = 0$ und $\lim\limits_{x \rightarrow x_0} g(x) = 0$ oder
	\item $\lim\limits_{x \rightarrow x_0} f(x) = \infty$ und $\lim\limits_{x \rightarrow x_0} g(x) = \infty$,
\end{itemize}
und falls $\exists \lim\limits_{x \rightarrow x_0} \frac{f'(x)}{g'(x)}$, so gilt:
\begin{equation*}
	\lim\limits_{x \rightarrow x_0} \frac{f(x)}{g(x)} = \lim\limits_{x \rightarrow x_0} \frac{f'(x)}{g'(x)}
\end{equation*}

\setcounter{subsection2}{\value{subsection}}
\section*{Konvexität und Jensen'sche Ungleichung}
\addcontentsline{toc}{subsection}{Konvexität und Jensen'sche Ungleichung}
\setcounter{subsection}{\value{subsection2}}

\addtocounter{subsection}{2}
\subsection{Definition zweimal und stetig differenzierbar}
Sei $f$ eine Funktion. Wir sagen, $f$ ist:
\begin{itemize}
	\item \textit{zweimal differenzierbar}, falls $f$ und $f'$ differenzierbar
	\item \textit{stetig differenzierbar}, falls $f$ differenzierbar und $f'$ stetig
	\item \textit{zweimal stetig differenzierbar}, falls $f$ zweimal differenzierbar und $f''$ stetig
\end{itemize}

\addtocounter{subsection}{3}
\subsection{Definition Konvexität}
Sei $I$ ein Interval und $f : I \rightarrow \mathbb{R}$.\\
Wenn $\forall x,y \in I$ und $\forall \alpha \in [0,1]$ gilt:
\begin{itemize}
	\item \makebox[2.5cm]{konvex,\hfill} falls $f(\alpha x + (1 - \alpha)y) \leq \alpha f(x) + (1 - \alpha)f(y)$
	\item \makebox[2.5cm]{streng konvex,\hfill} falls $f(\alpha x + (1 - \alpha)y) < \alpha f(x) + (1 - \alpha)f(y)$
	\item \makebox[2.5cm]{konkav,\hfill} falls $f(\alpha x + (1 - \alpha)y) \geq \alpha f(x) + (1 - \alpha)f(y)$
	\item \makebox[2.5cm]{streng konkav,\hfill} falls $f(\alpha x + (1 - \alpha)y) > \alpha f(x) + (1 - \alpha)f(y)$
\end{itemize}

\addtocounter{subsection}{-2}
\subsection{monoton wachsende Ableitung $\leftrightarrow$ konvex}
Sei $f: (a,b) \rightarrow \mathbb{R}$ in $(a,b)$ differenzierbar.\\
$f'$ ist in $(a,b)$ monoton wachsend gdw. $f$ konvex in $(a,b)$.

\stepcounter{subsection}
\subsection{zweite Ableitung $\rightarrow$ Konvexität}
Sei $f : (a,b) \rightarrow \mathbb{R}$ zweimal differenzierbar. Dann folgt:
\begin{itemize}
	\item $f''$ nicht negativ $\rightarrow$ $f$ ist konvex
	\item $f''$ positiv $\rightarrow$ $f$ ist streng konvex
	\item $f''$ nicht positiv $\rightarrow$ $f$ ist konkav
	\item $f''$ negativ $\rightarrow$ $f$ ist streng konkav
\end{itemize}

\subsection{Ungleichung auf Folge der Konvexität}
Sei $f : (a,b) \rightarrow \mathbb{R}$ in $(a,b)$ differenzierbar. \\
Dann gilt $\forall x_0, x_1 \in (a,b)$:
\begin{itemize}
	\item Falls $f$ konvex:
	\begin{equation*}
	f(x_0) + f'(x_0) \cdot (x_1 - x_0) \leq f(x_1)
	\end{equation*}
	\item Falls $f$ konkav:
	\begin{equation*}
	f(x_0) + f'(x_0) \cdot (x_1 - x_0) \geq f(x_1)
	\end{equation*}
	\item Wenn $f$ darüber hinaus streng konvex / streng konkav, so gilt:
	\begin{equation*}
	f(x_0) + f'(x_0) \cdot (x_1 - x_0) = f(x_1) \text{ gdw. } x_0 = x_1
	\end{equation*}
	
\end{itemize}

\subsection*{Jensen'sche Ungleichung}
\addcontentsline{toc}{subsection}{Jensen'sche Ungleichung}
Sei $f : (a,b) \rightarrow \mathbb{R}$, $n \geq 2$, seien $x_1, ..., x_n \in (a,b)$ und seien $p_1, ..., p_n > 0$ mit $\sum p_i = 1$.\\
Dann gilt:
\begin{itemize}
	\item Falls $f$ konvex:
	\begin{equation*}
	f\left(\sum_{k=1}^{n}p_kx_k\right) \leq \sum_{j=1}^{n}p_kf(x_k)
	\end{equation*}
	\item Falls $f$ konkav:
	\begin{equation*}
	f\left(\sum_{k=1}^{n}p_kx_k\right) \geq \sum_{j=1}^{n}p_kf(x_k)
	\end{equation*}
	\item Wenn $f$ streng konvex / streng konkav, so gilt:
	\begin{equation*}
	f\left(\sum_{k=1}^{n}p_kx_k\right) = \sum_{j=1}^{n}p_kf(x_k) \text{ gdw. } x_1 = x_2 = ... = x_n
	\end{equation*}
\end{itemize}

\subsection*{Ungleichung zwischen arithmetischem und geometrischem Mittel}
Seien $x_1,...,x_n > 0$ und seien $p_1, ..., p_n > 0$ mit $\sum p_i = 1$. Dann gilt:
\begin{equation*}
	\prod_{k=1}^{n}(x_k)^{p_k} \leq \sum_{k=1}^{n}p_kx_k
\end{equation*}

\newpage
\section{Integration}

\subsection*{Definition Unter- und Oberintegral}
Sei $f: [a,b] \rightarrow \mathbb{R}$ beschränkt.\\
Sei $Z := \{x_0, x_1, ..., x_n\}$ eine Zerlegung von $[a,b]$ mit den Grenzen $x_i$, wobei $x_0 < ... < x_n$.\\
Sei $m_i = \inf\limits_{x \in [x_{i-1}, x_i]} f(x)$ und $M_i = \sup\limits_{x \in [x_{i-1}, x_i]} f(x)$.\\
Definition Untersumme $U_Z(f)$ und Obersumme $O_Z(f)$:
\begin{align*}
	U_Z(f) &:= \int_{a}^{b}\varphi(x) \dif x & &\text{mit } \varphi(x) = m_i \text{ für } x \in (x_{i-1}, x_i) & &\\
	O_Z(f) &:= \int_{a}^{b}\psi(x) \dif x & &\text{mit } \psi(x) = M_i \text{ für } x \in (x_{i-1}, x_i) & &
\end{align*}
Wobei hier das Integral stufenweise berechnet werden kann, indem für jeden Teil der Zerlegung der Flächeninhalt des Rechtecks betrachtet wird.\\\\ 
Definition Unterintegral $U(f)$ und Oberintegral $O(f)$:
\begin{align*}
	U(f) &:= \sup\; \{U_Z(f) : Z \text{ ist Zerlegung von } [a,b]\}\\
	O(f) &:= \inf\; \{O_Z(f) : Z \text{ ist Zerlegung von } [a,b]\}
\end{align*}

\stepcounter{subsection}
\subsection{Definition Integral}
Sei $f: [a,b] \rightarrow \mathbb{R}$ beschränkt. \\
Falls $U(f) = O(f)$, so heißt $f$ \textit{integrierbar} und es wird definiert:
\begin{equation*}
	\int_{a}^{b} f(x) \dif x = U(f) = O(f)
\end{equation*}

\subsection{Integrierbarkeit stetiger Funktionen}
Jede stetige Funktion $f: [a,b] \rightarrow \mathbb{R}$ ist integrierbar.

\subsection{Integrierbarkeit monotoner Funktionen}
Jede monotone Funktion $f: [a,b] \rightarrow \mathbb{R}$ ist integrierbar.

\subsection{Alternative Definition Integrierbarkeit}
Sei $f: [a,b] \rightarrow \mathbb{R}$ beschränkt.\\
$f$ ist integrierbar gdw. $\forall \varepsilon > 0 : \exists Z$ Zerlegung, sodass:
\begin{equation*}
	O_Z(f) - U_Z(f) < \varepsilon
\end{equation*}

\subsection{gleichmäßige Stetigkeit}
Sei $f : [a,b] \rightarrow \mathbb{R}$ stetig. Es gilt:
\begin{equation*}
	\forall \varepsilon > 0 : \exists \delta > 0 : \forall x,y \in [a,b] \text{ mit } | x - y | < \delta : |f(x) - f(y)| < \varepsilon
\end{equation*}
 
\subsection{Eigenschaften und Rechenregeln des Integrals}
\begin{enumerate}
	\item \textbf{Linearität}\\
	Seien $f,g:[a,b] \rightarrow\mathbb{R}$ integrierbar und $\alpha, \beta \in \mathbb{R}$.\\
	Dann ist auch $\alpha f + \beta g$ integrierbar und
	\begin{equation*}
		\int_{a}^{b}\alpha f(x) + \beta g(x) \dif x = \alpha \int_{a}^bf(x)\dif x + \beta \int_a^bg(x)\dif x
	\end{equation*}
	\item \textbf{Monotonie}\\
	Seien $f,g:[a,b] \rightarrow\mathbb{R}$ integrierbar mit $\forall x \ in [a,b] : f(x) \leq g(x)$.\\
	Dann gilt:
	\begin{equation*}
		\int_a^bf(x)\dif x \leq \int_a^bg(x)\dif x
	\end{equation*}
	\item \textbf{Zerlegbarkeit}\\
	Sei $a < c < b$ und $f : [a,b] \rightarrow\mathbb{R}$.\\
	$f$ ist integrierbar auf $[a,b]$ gdw. $f$ auf $[a,c]$ und auf $[c,b]$ integrierbar.\\
	Es gilt dann:
	\begin{equation*}
		\int_a^bf(x)\dif x = \int_a^cf(x)\dif x + \int_c^bf(x)\dif x
	\end{equation*}
\end{enumerate}

\subsection{Integral mit \glqq verdrehtem\grqq{} Intervall}
Sei $a \leq b$ und $f: [a,b] \rightarrow \mathbb{R}$. Dann definieren wir:\\
\begin{equation*}
	\int_b^af(x)\dif x := - \int_a^bf(x)\dif x
\end{equation*}
Damit gilt die Zerlegbarkeit auch für alle $a,b,c \in \mathbb{R}$.\\\\
Aus der Definition folgt:
\begin{equation*}
	\int_{a}^{a}f(x) \dif x = 0
\end{equation*}
\end{document}
